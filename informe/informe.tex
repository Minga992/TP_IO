\documentclass[a4paper]{article}
\usepackage[spanish]{babel}
\usepackage[utf8]{inputenc}
\usepackage{fancyhdr}
\usepackage{charter}   % tipografía
\usepackage{graphicx}
\usepackage{makeidx}

\usepackage{float}
\usepackage{amsmath, amsthm, amssymb}
\usepackage{amsfonts}
\usepackage{sectsty}
\usepackage{wrapfig}
\usepackage{listings} % necesario para el resaltado de sintaxis
\usepackage{caption}
\usepackage{placeins}

\usepackage{hyperref} % agrega hipervínculos en cada entrada del índice
\hypersetup{          % (en el pdf)
    colorlinks=true,
    linktoc=all,
    citecolor=black,
    filecolor=black,
    linkcolor=black,
    urlcolor=black
}

\usepackage{color} % para snippets de código coloreados
\usepackage{fancybox}  % para el sbox de los snippets de código

\definecolor{litegrey}{gray}{0.94}

% \newenvironment{sidebar}{%
% 	\begin{Sbox}\begin{minipage}{.85\textwidth}}%
% 	{\end{minipage}\end{Sbox}%
% 		\begin{center}\setlength{\fboxsep}{6pt}%
% 		\shadowbox{\TheSbox}\end{center}}
% \newenvironment{warning}{%
% 	\begin{Sbox}\begin{minipage}{.85\textwidth}\sffamily\lite\small\RaggedRight}%
% 	{\end{minipage}\end{Sbox}%
% 		\begin{center}\setlength{\fboxsep}{6pt}%
% 		\colorbox{litegrey}{\TheSbox}\end{center}}

\newenvironment{codesnippet}{%
	\begin{Sbox}\begin{minipage}{\textwidth}\sffamily\small}%
	{\end{minipage}\end{Sbox}%
		\begin{center}%
		\colorbox{litegrey}{\TheSbox}\end{center}}



\usepackage{fancyhdr}
\pagestyle{fancy}

%\renewcommand{\chaptermark}[1]{\markboth{#1}{}}
\renewcommand{\sectionmark}[1]{\markright{\thesection\ - #1}}

\fancyhf{}

\fancyhead[LO]{Sección \rightmark} % \thesection\
\fancyfoot[LO]{\small{Confalonieri, Mignanelli}}
\fancyfoot[RO]{\thepage}
\renewcommand{\headrulewidth}{0.5pt}
\renewcommand{\footrulewidth}{0.5pt}
\setlength{\hoffset}{-0.8in}
\setlength{\textwidth}{16cm}
%\setlength{\hoffset}{-1.1cm}
%\setlength{\textwidth}{16cm}
\setlength{\headsep}{0.5cm}
\setlength{\textheight}{25cm}
\setlength{\voffset}{-0.7in}
\setlength{\headwidth}{\textwidth}
\setlength{\headheight}{13.1pt}

\renewcommand{\baselinestretch}{1.1}  % line spacing


\usepackage{underscore}
\usepackage{caratula}
\usepackage{url}
\usepackage{color}
\usepackage{clrscode3e} % necesario para el pseudocodigo (estilo Cormen)




\begin{document}

\lstset{
  language=C++,                    % (cambiar al lenguaje correspondiente)
  backgroundcolor=\color{white},   % choose the background color
  basicstyle=\footnotesize,        % size of fonts used for the code
  breaklines=true,                 % automatic line breaking only at whitespace
  captionpos=b,                    % sets the caption-position to bottom
  commentstyle=\color{red},    % comment style
  escapeinside={\%*}{*)},          % if you want to add LaTeX within your code
  keywordstyle=\color{blue},       % keyword style
  stringstyle=\color{blue},     % string literal style
}

\thispagestyle{empty}
\materia{Investigación operativa}
\submateria{Segundo Cuatrimestre de 2015}
\titulo{Coloreo Particionado de Grafos}
\subtitulo{Planos de Corte (INSERTE MEJORAS)}
\integrante{Confalonieri, Gisela Belén}{511/11}{gise_5291@yahoo.com.ar} % por cada integrante (apellido, nombre) (n° libreta) (e-mail)
\integrante{Mignanelli, Alejandro Rubén}{609/11}{minga_titere@hotmail.com} 

\maketitle
\newpage

\thispagestyle{empty}
\vfill
%\begin{abstract}
%    \vspace{0.5cm}
%	
%
%\end{abstract}

\thispagestyle{empty}
\vspace{1.5cm}
\tableofcontents
\newpage

%\normalsize
 
\newpage

\section{Introducción}

En el presente trabajo se realizará un estudio comparativo entre las estrategias {\it Branch and Bound} y {\it Cut and Branch} sobre el problema del coloreo particionado de grafos (que se explicará en otra sección de este informe) al encararlo desde Programación Lineal Entera.

Los objetivos del trabajo son los siguientes:

\begin{itemize}

	\item {\bf Interacción con CPLEX:}
	
	CPLEX es un paquete de software comercial y académico para resolución de problemas de Programación Lineal y Programación Lineal Entera. Uno de los fines de nuestro trabajo consiste entonces en aprender a usar este paquete tanto resolviendo problemas, como reemplazando parte del trabajo que éste realiza con código propio.
	
	\item {\bf Modelado de un PLEM:}
	
	El modelado de un PLEM (Problema de Programación Lineal Entero Mixto) cumple un rol muy importante. Dado que PLEM es NP-difícil, es importante ajustar lo mejor posible el conjunto de soluciones factibles, de manera de tener menos "opciones" que recorrer.  Por ejemplo, en el caso de coloreo de grafos existe el problema de que tiene muchas soluciones simétricas para un mismo grafo, que al fin y al cabo representan una misma solución.  Por ello, parte del objetivo de este trabajo es no sólo realizar un PLEM correcto, sino elaborar estrategias que permitan ver la menor cantidad de soluciones posibles.
	
	\item {\bf Desigualdades válidas e implementación de planos de corte:}
	
	Otro de los propósitos de este trabajo es lograr un mayor entendimiento de qué son y cómo funcionan los planos de corte. Si bien en la materia los estudiamos a nivel teórico, es importante toparnos con ellos a nivel práctico e intentar implementarlos. Para esto, utilizaremos las desigualdades válidas {\it clique} y {\it odd-hole}.

	\item {\bf Comparación entre distintos métodos de resolución:}
	
	La última meta será realizar una comparación entre algunos métodos de resolución general:
	
	\begin{itemize}
	
		\item {\it Branch-and-Bound:}
		
		Esta técnica subdivide el problema sucesivamente en otros más pequeños, eliminando ciertas soluciones fraccionarias, y manteniendo durante el recorrido del árbol generado una cota superior y otra
inferior para el óptimo buscado. Es la resolución automática de CPLEX, por lo que le quitaremos todo el preprocesamiento y cortes que este paquete le añade, con el fin de obtener un resultado basado solamente en Branch-and-Bound.
		
		\item {\it Cut-and-Branch:}
		
		La idea es añadir cortes en el nodo raíz, para luego realizar un Branch-and-Bound clásico, que en teoría debería tomar menos tiempo dado que la formulación del PLEM es más restrictiva. Sin embargo, procesar un problema con demasiadas restricciones también puede empeorar la performance, por lo cual un subobjetivo en esta etapa será encontrar una cantidad apropiada de cortes a agregar, de manera que favorezca la resolución.
	
	\end{itemize}
	
	
\end{itemize}

\newpage

\section{Problema del Coloreo Particionado de Grafos}

El problema de coloreo de grafos ha sido ampliamente estudiado y aparece en numerosas aplicaciones de la vida real, como por ejemplo en problemas de scheduling, asignación de frecuencias, secuenciamiento, etc. Formalmente, el problema puede ser definido de la siguiente forma: 

Dado un grafo $G = (V,E)$ con $n = |V|$ vértices y $m = |E|$ aristas, un coloreo de $G$ consiste en una asignación de colores o etiquetas a cada vértice $p\in V$ de forma tal que todo par de vértices $(p, q) \in E$ poseen colores distintos. El problema de coloreo de grafos consiste en encontrar un coloreo que utilice la menor cantidad posible de colores distintos.

A partir de diferentes aplicaciones, surgieron variaciones o generalizaciones de este problema, como el problema de coloreo particionado de grafos. En este problema el conjunto $V$ se encuentran dividido en una partición $V_1 , . . . , V_k$ , y el objetivo es asignar un color a sólo un vértice de cada partición, de manera tal que dos vértices adyacentes no reciban el mismo color, minimizando la cantidad de colores utilizados.

\newpage

\section{Modelado del problema}

Sea un grafo $G = (V,E)$, con $n = |V|$.  Definimos las siguientes variables binarias:

\begin{itemize}
	\item Para $i = 1,...,n, j = 1,...,n$	
	\begin{equation*}
	x_{ij} = \begin{cases}
				1 & \mbox{si el nodo }i\mbox{ es pintado con el color }j\\
				0 & \mbox{caso contrario}
			\end{cases}
	\end{equation*}
	
	\item Para $j = 1,..,n$
	\begin{equation*}
	w_j = \begin{cases}
			1 & \mbox{si algun nodo es pintado con el color }j\\
			0 & \mbox{caso contrario}
		\end{cases}
	\end{equation*}
\end{itemize}

Sea $C$ el conjunto de colores posibles de utilizar.  Buscamos minimizar la cantidad de colores distintos usados:

\begin{equation*}
min \sum_{j\in C} w_j
\end{equation*}

Sujeto a:

\begin{itemize}
	\item La variable $w_j$ es verdadera sii algún vértice usa el color j:
	\begin{equation*}
	x_{ij} \leq w_j \quad \forall j \in C, \forall i \in V
	\end{equation*}
	
	\item Dos vecinos no pueden usar el mismo color:
	\begin{equation*}
	x_{ij}+x_{kj} \leq 1 \quad \forall j \in C, \forall (i,k) \in E
	\end{equation*}
	
	\item Cada conjunto $p$ de la partición $P$ tiene exactamente un color asignado:
	\begin{equation*}
	\sum_{x_i \in p} \sum_{j \in C} x_{ij} = 1 \quad \forall i \in V, p \in P
	\end{equation*}
	
	\item No se permite usar un color hasta que no se hayan usado todos los anteriores:
	\begin{equation*}
	w_j \geq w_{j+1} \quad \forall 1 \leq j < |C|
	\end{equation*}
	
	\item Ninguna partición puede estar coloreada con un color de etiqueta mayor a su índice:
	\begin{equation*}
	x_{ij} = 0 \quad \forall j > p(i)+1
	\end{equation*}
\end{itemize}

\newpage

\section{Desigualdades válidas}

A continuación presentaremos dos familias desigualdades y demostraremos que son válidas para el problema de coloreo particionado.

\paragraph{Desigualdad Clique:} Sea $j_0$ $\in$ $ \{ 1,...,n \} $ y sea $K$ una clique maximal de $G$. La desigualdad clique está definida por:

\begin{equation} \label{eq:des1}
\sum_{p \in K} x_{pj_0} \leq w_{j_0}
\end{equation}

{\it Demostración:}

Queremos ver que es desigualdad válida.  Es decir, que toda solución factible del PLEM planteado cumple (\ref{eq:des1}).

Supongamos que no, o sea que $\exists s$ solución factible tal que:

\begin{equation*}
\sum_{p \in K} x_{pj_0} > w_{j_0}
\end{equation*}

\begin{itemize}
	\item Si $w_{j_0} = 0$, no cumple con la restricción $x_{ij} \leq w_j$: {\it ABSURDO}.
	
	\item Si $w_{j_0} = 1$ entonces
	
	\begin{equation*}
	\sum_{p \in K} x_{pj_0} > 1
	\end{equation*}

	O sea que hay al menos dos nodos en $K$ con color $j_0$, lo que significa que $\exists i,h \in K$ tal que $x_{ij_0} = 1$ y $x_{hj_0} =1$.  Pero como $K$ es una clique, $(i,h) \in E$, así que no se cumple la restricción $x_{ij} + x_{kj} \leq 1 \quad \forall j \in C,\forall (i,k) \in E$: {\it ABSURDO}.
\end{itemize}

Estos absurdos provienen de suponer que existe $s$ solución factible que no cumple con la desigualdad de clique.  Por lo tanto, la desigualdad resulta válida para toda solución factible del PLEM.


\paragraph{Desigualdad Odd-hole:} Sea $j_0 \in \{ 1,...,n \}$ y sea $C_{2k+1} = v_1 ,..., v_{2k+1}, k \geq 2$, un agujero de longitud impar. La desigualdad odd-hole está definida por:

\begin{equation} \label{eq:des2}
\sum_{p \in C_{2k+1}} x_{pj_0} \leq kw_{j_0}
\end{equation}

{\it Demostración:}

Queremos ver que es desigualdad válida.   Es decir, que toda solución factible del PLEM planteado cumple (\ref{eq:des2}).

Supongamos que no, o sea que $\exists s$ solución factible tal que:

\begin{equation*}
\sum_{p \in C_{2k+1}} x_{pj_0} > kw_{j_0}
\end{equation*}

\begin{itemize}

	\item Si $w_{j_0} = 0$, no cumple con la restricción $x_{ij} - w_j \leq 0$: {\it ABSURDO}.
	
	\item Si $w_{j_0} = 1$ entonces
	
	\begin{equation*}
	\sum_{p \in C_{2k+1}} x_{pj_0} > k
	\end{equation*}

	O sea que hay al menos $k+1$ nodos en $C_{2k+1}$ con color $j_0$. Por el {\bf Lema 1}, esto significa que $\exists i,h \in C_{2k+1}$ tal que $x_{ij_0} =1$ y $x_{hj_0} =1$ con $(i,h) \in E$. Pero esto no cumple la restricción $x_{ij} + x_{kj} \leq 1 \quad \forall j \in C,\forall (i,k) \in E$: {\it ABSURDO}.
\end{itemize}

Estos absurdos provienen de suponer que existe $s$ solución factible que no cumple con la desigualdad de agujero impar.  Por lo tanto, la desigualdad resulta válida para toda solución factible del PLEM.

\paragraph{Lema 1:}

Sea $C_{2k+1}$ circuito impar $(k \geq 2)$ y $H$ subconjunto de nodos de $C_{2k+1}$ tal que $|H| \geq k+1$. Entonces $\exists v,w \in H$ tal que $(v,w) \in E(C_{2k+1})$.

{\it Demostración:}

Supongamos que no. O sea que existe un $H$ subconjunto de nodos de $C_{2k+1}$ tal que $|H| \geq k+1$ y tal que no existe $v,w \in H$ tal que $(v,w) \in E(C_{2k+1})$. Pero si esto es cierto, significa que el grafo inducido por $H$(no recuerdo si esta es la definición pero se entiende la idea, corregir(ALE)) es un grafo sin aristas. Eso significa que existen al menos $k+1$ nodos que pertenecen a $C_{2k+1}$ pero no a $H$. No se como cerrar la idea, pero esta ahiiii!!(ALE).


\newpage

\section{Experimentación}

\newpage

\section{Conclusiones y trabajo futuro} 

\end{document}
