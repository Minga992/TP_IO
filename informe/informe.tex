\documentclass[a4paper]{article}
\usepackage[spanish]{babel}
\usepackage[utf8]{inputenc}
\usepackage{fancyhdr}
\usepackage{charter}   % tipografía
\usepackage{graphicx}
\usepackage{makeidx}

\usepackage{float}
\usepackage{amsmath, amsthm, amssymb}
\usepackage{amsfonts}
\usepackage{sectsty}
\usepackage{wrapfig}
\usepackage{listings} % necesario para el resaltado de sintaxis
\usepackage{caption}
\usepackage{placeins}

\usepackage{hyperref} % agrega hipervínculos en cada entrada del índice
\hypersetup{          % (en el pdf)
    colorlinks=true,
    linktoc=all,
    citecolor=black,
    filecolor=black,
    linkcolor=black,
    urlcolor=black
}

\usepackage{color} % para snippets de código coloreados
\usepackage{fancybox}  % para el sbox de los snippets de código

\definecolor{litegrey}{gray}{0.94}

% \newenvironment{sidebar}{%
% 	\begin{Sbox}\begin{minipage}{.85\textwidth}}%
% 	{\end{minipage}\end{Sbox}%
% 		\begin{center}\setlength{\fboxsep}{6pt}%
% 		\shadowbox{\TheSbox}\end{center}}
% \newenvironment{warning}{%
% 	\begin{Sbox}\begin{minipage}{.85\textwidth}\sffamily\lite\small\RaggedRight}%
% 	{\end{minipage}\end{Sbox}%
% 		\begin{center}\setlength{\fboxsep}{6pt}%
% 		\colorbox{litegrey}{\TheSbox}\end{center}}

\newenvironment{codesnippet}{%
	\begin{Sbox}\begin{minipage}{\textwidth}\sffamily\small}%
	{\end{minipage}\end{Sbox}%
		\begin{center}%
		\colorbox{litegrey}{\TheSbox}\end{center}}



\usepackage{fancyhdr}
\pagestyle{fancy}

%\renewcommand{\chaptermark}[1]{\markboth{#1}{}}
\renewcommand{\sectionmark}[1]{\markright{\thesection\ - #1}}

\fancyhf{}

\fancyhead[LO]{Sección \rightmark} % \thesection\
\fancyfoot[LO]{\small{Confalonieri, Mignanelli}}
\fancyfoot[RO]{\thepage}
\renewcommand{\headrulewidth}{0.5pt}
\renewcommand{\footrulewidth}{0.5pt}
\setlength{\hoffset}{-0.8in}
\setlength{\textwidth}{16cm}
%\setlength{\hoffset}{-1.1cm}
%\setlength{\textwidth}{16cm}
\setlength{\headsep}{0.5cm}
\setlength{\textheight}{25cm}
\setlength{\voffset}{-0.7in}
\setlength{\headwidth}{\textwidth}
\setlength{\headheight}{13.1pt}

\renewcommand{\baselinestretch}{1.1}  % line spacing


\usepackage{underscore}
\usepackage{caratula}
\usepackage{url}
\usepackage{color}
\usepackage{clrscode3e} % necesario para el pseudocodigo (estilo Cormen)




\begin{document}

\lstset{
  language=C++,                    % (cambiar al lenguaje correspondiente)
  backgroundcolor=\color{white},   % choose the background color
  basicstyle=\footnotesize,        % size of fonts used for the code
  breaklines=true,                 % automatic line breaking only at whitespace
  captionpos=b,                    % sets the caption-position to bottom
  commentstyle=\color{red},    % comment style
  escapeinside={\%*}{*)},          % if you want to add LaTeX within your code
  keywordstyle=\color{blue},       % keyword style
  stringstyle=\color{blue},     % string literal style
}

\thispagestyle{empty}
\materia{Investigación operativa}
\submateria{Segundo Cuatrimestre de 2015}
\titulo{Investigación operativa(hay que esforzarse más en esto)}
%\subtitulo{Subtítulo}
\integrante{Confalonieri, Gisela Belén}{511/11}{gise_5291@yahoo.com.ar} % por cada integrante (apellido, nombre) (n° libreta) (e-mail)
\integrante{Mignanelli, Alejandro Rubén}{609/11}{minga_titere@hotmail.com} 

\maketitle
\newpage

\thispagestyle{empty}
\vfill
%\begin{abstract}
%    \vspace{0.5cm}
%	
%
%\end{abstract}

\thispagestyle{empty}
\vspace{1.5cm}
\tableofcontents
\newpage

%\normalsize
 
\newpage

%\section{Ejercicio 1}

\section{Introducción}

\newpage

\section{Modelado del problema}

\newpage

\section{Código}

\newpage

\section{Experimentación}

\newpage

\section{Conclusiones y trabajo futuro} 

\end{document}
