\documentclass[a4paper]{article}
\usepackage[spanish]{babel}
\usepackage[utf8]{inputenc}
\usepackage{fancyhdr}
\usepackage{charter}   % tipografía
\usepackage{graphicx}
\usepackage{makeidx}

\usepackage{float}
\usepackage{amsmath, amsthm, amssymb}
\usepackage{amsfonts}
\usepackage{sectsty}
\usepackage{wrapfig}
\usepackage{listings} % necesario para el resaltado de sintaxis
\usepackage{caption}
\usepackage{placeins}

\usepackage{hyperref} % agrega hipervínculos en cada entrada del índice
\hypersetup{          % (en el pdf)
    colorlinks=true,
    linktoc=all,
    citecolor=black,
    filecolor=black,
    linkcolor=black,
    urlcolor=black
}

\usepackage{color} % para snippets de código coloreados
\usepackage{fancybox}  % para el sbox de los snippets de código

\definecolor{litegrey}{gray}{0.94}

% \newenvironment{sidebar}{%
% 	\begin{Sbox}\begin{minipage}{.85\textwidth}}%
% 	{\end{minipage}\end{Sbox}%
% 		\begin{center}\setlength{\fboxsep}{6pt}%
% 		\shadowbox{\TheSbox}\end{center}}
% \newenvironment{warning}{%
% 	\begin{Sbox}\begin{minipage}{.85\textwidth}\sffamily\lite\small\RaggedRight}%
% 	{\end{minipage}\end{Sbox}%
% 		\begin{center}\setlength{\fboxsep}{6pt}%
% 		\colorbox{litegrey}{\TheSbox}\end{center}}

\newenvironment{codesnippet}{%
	\begin{Sbox}\begin{minipage}{\textwidth}\sffamily\small}%
	{\end{minipage}\end{Sbox}%
		\begin{center}%
		\colorbox{litegrey}{\TheSbox}\end{center}}



\usepackage{fancyhdr}
\pagestyle{fancy}

%\renewcommand{\chaptermark}[1]{\markboth{#1}{}}
\renewcommand{\sectionmark}[1]{\markright{\thesection\ - #1}}

\fancyhf{}

\fancyhead[LO]{Sección \rightmark} % \thesection\
\fancyfoot[LO]{\small{Confalonieri, Mignanelli}}
\fancyfoot[RO]{\thepage}
\renewcommand{\headrulewidth}{0.5pt}
\renewcommand{\footrulewidth}{0.5pt}
\setlength{\hoffset}{-0.8in}
\setlength{\textwidth}{16cm}
%\setlength{\hoffset}{-1.1cm}
%\setlength{\textwidth}{16cm}
\setlength{\headsep}{0.5cm}
\setlength{\textheight}{25cm}
\setlength{\voffset}{-0.7in}
\setlength{\headwidth}{\textwidth}
\setlength{\headheight}{13.1pt}

\renewcommand{\baselinestretch}{1.1}  % line spacing


\usepackage{underscore}
\usepackage{caratula}
\usepackage{url}
\usepackage{color}
\usepackage{clrscode3e} % necesario para el pseudocodigo (estilo Cormen)




\begin{document}

\lstset{
  language=C++,                    % (cambiar al lenguaje correspondiente)
  backgroundcolor=\color{white},   % choose the background color
  basicstyle=\footnotesize,        % size of fonts used for the code
  breaklines=true,                 % automatic line breaking only at whitespace
  captionpos=b,                    % sets the caption-position to bottom
  commentstyle=\color{red},    % comment style
  escapeinside={\%*}{*)},          % if you want to add LaTeX within your code
  keywordstyle=\color{blue},       % keyword style
  stringstyle=\color{blue},     % string literal style
}

\thispagestyle{empty}
\materia{Investigación operativa}
\submateria{Segundo Cuatrimestre de 2015}
\titulo{Investigación operativa(hay que esforzarse más en esto)}
%\subtitulo{Subtítulo}
\integrante{Confalonieri, Gisela Belén}{511/11}{gise_5291@yahoo.com.ar} % por cada integrante (apellido, nombre) (n° libreta) (e-mail)
\integrante{Mignanelli, Alejandro Rubén}{609/11}{minga_titere@hotmail.com} 

\maketitle
\newpage

\thispagestyle{empty}
\vfill
%\begin{abstract}
%    \vspace{0.5cm}
%	
%
%\end{abstract}

\thispagestyle{empty}
\vspace{1.5cm}
\tableofcontents
\newpage

%\normalsize
 
\newpage

\section{Introducción}

En el presente trabajo se realizará un estudio comparativo entre las estrategias Branch and Bound y Cut and Branch sobre el problema del coloreo particionado de grafos(que se explicará en otra sección de este informe) encarado desde Programación Lineal Entera.
Los objetivos del trabajo son los siguientes:

\begin{itemize}

	\item Interacción con CPLEX:
	
	CPLEX es un paquete de software comercial y académico para resolución de problemas de Programación Lineal y Programación Lineal Entera. Uno de los fines de nuestro trabajo consiste entonces en aprender a usar este paquete tanto resolviendo problemas, como reemplazando parte del trabajo que este realiza con código propio.
	
	\item Modelado de un PLE:
	
	El modelado de un PLEM cumple un rol fundamental, ya que como en general se modelan problemas cuyo mejor algoritmo es exponencial, es importante que el modelo no solo represente correctamente el problema, sino que permita observar la menor cantidad de soluciones posibles(no se como explicar bien esto(ALE)). Por ello, parte del objetivo de este trabajo es no solo realizar un PLEM correcto, sino elaborar estrategias que permitan ver la menor cantidad de sol posibles.
	
	\item Desigualdades válidas, cortes e implementación de estos:
	
	Otro de los propósitos de este trabajo es lograr un mayor entendimiento de que son y como funcionan los cortes. Si bien en la materia los estudiamos a nivel teórico, es importante toparnos con ellos a nivel práctico e intentar implementarlos. Para esto, utilizaremos las desigualdades válidas clique y odd-hole.

	\item Comparación entre distintos métodos de resolución
	
	La última meta será realizar una comparación entre algunos métodos de resolución general:
	\begin{itemize}
	
		\item Branch-and-Bound:
		
		(Breve explicación de que es branch and bound(ALE)). Es la resolución automática de CPLEX, por lo que le quitaremos todo el preprocesamiento y cortes que este paquete le añade, con el fin de obtener un resultado basado solamente en Branch-and-Bound 
		
		\item Cut-and-Branch:
		
		La idea es añadir cortes en el nodo raíz, para luego realizar un Branch-and-Bound clásico, que en teoría debería tomar menos tiempo dado que la formulación del PLE es más restrictiva. Sin embargo, dado que agregar un sinfin de restricciones también provoca una demora, un subobjetivo en esta etapa será encontrar una cantidad apropiada de cortes de manera que se acorte el branch and bound, pero que el preprocesamiento no tome demasiado(lo escribi como el orto(ALE)).
	
	\end{itemize}
	
	
\end{itemize}

\newpage

\section{Problema del Coloreo Particionado de Grafos}

El problema de coloreo de grafos ha sido ampliamente estudiado y aparece en numerosas aplicaciones de la vida real, como por ejemplo en problemas de scheduling, asignación de frecuencias,secuenciamiento, etc. Formalmente, el problema puede ser definido de la siguiente forma: 

Dado un grafo $G = (V,E)$ con $n = |V|$ vértices y $m = |E|$ aristas, un coloreo de $G$ consiste en una asignación de colores o etiquetas a cada vértice $p\in V$ de forma tal que todo par de vértices $(p, q) \in E$ poseen colores distintos. El problema de coloreo de grafos consiste en encontrar un coloreo que utilice la menor cantidad posible de colores distintos.

A partir de diferentes aplicaciones, surgieron variaciones o generalizaciones de este problema, como el problema de coloreo particionado de grafos. En este problema el conjunto $V$ se encuentran dividido en una partición $V_1 , . . . , V_k$ , y el objetivo es asignar un color a sólo un vértice de cada partición, de manera tal que dos vértices adyacentes no reciban el mismo color, minimizando la cantidad de colores utilizados.

\newpage

\section{Modelado del problema}


Variables:

$G$ : Grafo

$V$ : Conjunto de los vértices del $G$

$E$ : Conjunto de las aristas de $G$

$C$ : Conjunto de colores

$P$ : Conjuntos de vértices agrupados según la partición

$x_{ij}$ : es verdadera sii el vértice $i$ es coloreado con el color $j$

$w_j$ :es verdadera sii el color $j$ fue usado

PLEM:

\begin{tabbing}

\hspace*{4cm} \= \hspace*{4cm} \kill

$Min$ $ \sum_{j\in C} w_j$ \> \\

Sujeto a:\> \\

$x_{ij} - w_j$ $\leq$ $0$\>   $\forall j \in C$, $\forall i \in V$\\

$x_{ij}$ $+$ $x_{kj}$ $\leq$ $1$ \>  $\forall j \in C$, $\forall (i,k) \in E$\\

$\sum_{x_i \in P} \sum_{j \in C} x_{ij}$ $=$ $1$ \> $\forall i \in V$, $\forall p \in P$\\

$w_j$ $-$ $w_{j+1}$ $\geq$ $0$\> \\

$x_{ij}$ $=$ $0$ \> \\

\end{tabbing}

$Min$ $ \sum_{j\in C} w_j$

Sujeto a:

$x_{ij} - w_j$ $\leq$ $0$   $\forall j \in C$, $\forall i \in V$

$x_{ij}$ $+$ $x_{kj}$ $\leq$ $1$   $\forall j \in C$, $\forall (i,k) \in E$

$\sum_{x_i \in P} \sum_{j \in C} x_{ij}$ $=$ $1$  $\forall i \in V$, $\forall p \in P$

$w_j$ $-$ $w_{j+1}$ $\geq$ $0$

$x_{ij}$ $=$ $0$


\newpage

\section{Código}

ESTO NO VA ACA, pero voy tirando la demo. Después hay que acomodarlo:

\paragraph{Desigualdad Clique:} Sea $j_0$ $\in$ $ \{ 1,...,n \} $ y sea $K$ una clique maximal de $G$. La desigualdad clique está definida por:

\begin{equation} \label{eq:des1}
\sum_{p \in K} x_{pj_0} \leq w_{j_0}
\end{equation}

Demo:

Queremos ver que es desigualdad válida. O sea, que se cumple (\ref{eq:des1}) $\forall$ solución factible.

Supongamos que no, o sea que $\exists$ $s$ solución factible tal que:

\begin{equation*}
\sum_{p \in K} x_{pj_0} > w_{j_0}
\end{equation*}

\begin{itemize}

	\item Si $w_{j_0} = 0$, ABSURDO, porque no cumple con la restricción $x_{ij} - w_j$ $\leq$ $0$.
	
	\item Si $w_{j_0} = 1$ entonces
	
	\begin{equation*}
	\sum_{p \in K} x_{pj_0} > 1
	\end{equation*}

	O sea que hay al menos dos nodos en $K$ con color $j_0$, lo que significa que $\exists$ $i,h \in K$ tal que $x_{ij_0} =1$ y $x_{hj_0} =1$.
	ABSURDO, puesto que al $K$ ser clique, $(i,h) \in X$ y no se cumple la restricción $x_{ij}$ $+$ $x_{kj}$ $\leq$ $1$   $\forall j \in C$, $\forall (i,k) \in E$
\end{itemize}

Estos absurdos provienen de suponer que existe $s$, por lo que la desigualdad válida es correcta.


\paragraph{Desigualdad Odd-hole:} Sea $j_0$ $\in$ $ \{ 1,...,n \} $ y sea $C_{2k+1} = v_1 ,..., v_{2k+1}$, $k \geq 2$ , un agujero de longitud impar. La desigualdad odd-hole está definida por:

\begin{equation} \label{eq:des2}
\sum_{p \in C_{2k+1}} x_{pj_0} \leq kw_{j_0}
\end{equation}

Demo:

Queremos ver que es desigualdad válida. O sea, que se cumple (\ref{eq:des2}) $\forall$ solución factible.

Supongamos que no, o sea que $\exists$ $s$ solución factible tal que:

\begin{equation*}
\sum_{p \in C_{2k+1}} x_{pj_0} > kw_{j_0}
\end{equation*}

\begin{itemize}

	\item Si $w_{j_0} = 0$, ABSURDO, porque no cumple con la restricción $x_{ij} - w_j$ $\leq$ $0$.
	
	\item Si $w_{j_0} = 1$ entonces
	
	\begin{equation*}
	\sum_{p \in C_{2k+1}} x_{pj_0} > k
	\end{equation*}

	O sea que hay al menos $k+1$ nodos en $C_{2k+1}$ con color $j_0$, lo que significa que por LEMA $\exists$ $i,h \in C_{2k+1}$ tal que $x_{ij_0} =1$ y $x_{hj_0} =1$ con $(i,h) \in X$.
	ABSURDO, pues no se cumple la restricción $x_{ij}$ $+$ $x_{kj}$ $\leq$ $1$   $\forall j \in C$, $\forall (i,k) \in E$
\end{itemize}

Estos absurdos provienen de suponer que existe $s$, por lo que la desigualdad válida es correcta.

LEMA1:

Sea $C_{2k+1}$ circuito impar $(k\geq 2)$ y $H$ subconjunto de nodos de $C_{2k+1}$ tal que $\# H \geq k+1$ entonces $\exists v,w \in H$ tal que $(v,w) \in X(C_{2k+1})$.

Demo:

Supongamos que no. O sea que existe un $H$ subconjunto de nodos de $C_{2k+1}$ tal que $\# H \geq k+1$ pero sin embargo no existe $v,w \in H$ tal que $(v,w) \in X(C_{2k+1})$. Pero si esto es cierto, significa que el grafo inducido por $H$(no recuerdo si esta es la definición pero se entiende la idea, corregir(ALE)) es un grafo sin aristas. Eso significa que existen al menos $k+1$ nodos que pertenecen a $C_{2k+1}$ pero no a $H$. No se como cerrar la idea, pero esta ahiiii!!(ALE).


\newpage

\section{Experimentación}

\newpage

\section{Conclusiones y trabajo futuro} 

\end{document}
